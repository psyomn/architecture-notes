\section{Analysis}

Ultimately we want two things. A game engine, and a game editor. The engine is 
not to be modified - a standalone binary, or the like, must be provided, that 
will look in predetermined locations for assets and tie the whole game
together.

Assets contain anything from text, graphics and sounds. Assets can be organized
in subcategories as well, since for example, in games we would like to discern
between background music, and sound effects. This organization is purely
aesthetic, and to help the user navigate through many things. Assets should 
contain the following:

\begin{itemize}
\item Graphics - Anything that has to do with things being drawn on screen.
Tiles, Tilesets, portraits, backgrounds, etc.
\item Sound - Anything that causes noise. Background music, sound effects, and
anything else that falls into this category.
\item Text - Anything that can be written down in a natural language.
\item Maps - A composite that brings many of the aforementioned assets
together, to make up the virtual world.
\end{itemize}

\subsection{Graphics}

In the respect of graphics, we want Haze to be a retro game engine. So in this
respect we will be interested mostly with the manipulation of 2d graphic files.
However blending in 3d aspects to 2d environments is totally possible and could
prove to be desirable by the userbase in the future.

For the graphics we require a library that can handle image manipulation. The
most important aspect being croping as we want to extract different tiles from
the tilesets that will exist in the assets. Just a simple reminder on some
jargon:

\paragraph{Tileset}: is an image file that contains smaller images that can be
used repeatedly as a pattern. 

\paragraph{Tile}: is a smaller graphic that is a pattern, that can be used to 
portray repeated graphics.

\subsection{Text}

Text is more important in role playing games as they traditionally are games
with richer storylines. Therefore it is even greater motivation to detect this
and prepare a proper mechanism.

A mechanism that has worked in the past in several applications I had to work
with and proved to be the right path to take in development, is
\textit{internationalization} and \textit{localization} of the system. Those 
are big words for simply meaning, storing the text of the application separately
and telling the system to look for an \textit{id} that containts the wanted 
text in that moment. 

A simple example would involve a menu asking your name. There would be a label
with the caption ``name''. For English speakers, that would be fine. But what
if someone wishes to use the application who does not know how to communicate
in that language? Should we make a copy of the project, and replace all the 
string instances with ``name'' and others to that other language? Or should we
factor out this functionality in some other way?

That is exactly where the said mechanisms come in, and we can have multiple
translations for the same application available, ready to update, and even add
more languages without the need of reloading the application. In the future if
let's say a game is made by this editor and engine, and it is accepted, we
would like to provide a means for translators to have no trouble to go about
their work.

\subsection{Sound}

Sound is pretty much anything that makes noise while the player is playing the
game. If more recent technology is to be used in order to write the music, then
I would suggest the use of Ogg Vorbis. The engine is open source and Ogg Vorbis
is pretty awesome when it comes to application.

For background music, we want to be able to loop the song. It would be also 
quite adventageous to point exactly where we want it to loop, as a song might
have an intro that is not repeated later on. I've seen this being omitted a few
many times and it's something that I personally did not like when I found it 
lacking.

Sound effects should be possible to mess around with effects. For example 
switching the pitch slightly for a repeating sound might make it sound less 
monotonous and irritate the user as well. 

Also, another thing to take into consideration is voice actors, and games that
not only have the text of speech displayed, but actually can enabled it to have
the speech spoken out. This is another aspect of internationalization and 
localization that we need to take care off later on as well.

\subsection{Maps}

Maps are bits and pieces that build the world of the game. They are pretty
essential, and the correct analysis, design and implementation of them, in my
opinion, is critical. There's a lot of ways the maps can be expressed in a 
system, but many of these ways can be considered wrong.

When we boil it down to basics, there's two aspects we really need to look out
for. One of them are moving entities, and the others stills. A quality one has 
to deal with is also collisions, and appropriate actions to take on these 
collisions. Role playing games are very simple in this aspect (but we should
still provide a rich set of actions that may be performed for the sake of 
expressiveness of our users).

\subsection{Proposed Directory Structure for Assets}

\begin{lstlisting}
  /items.xml
  /graphics/tilesets
  /graphics/portraits
  /sounds/music
  /sounds/sfx
  /sounds/speech/en
  /sounds/speech/el - Example, a Hellenic voice translation
  /lang/en
  /lang/el - Example, a Hellenic translation
\end{lstlisting}

