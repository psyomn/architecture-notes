\section{Abstract}

Game editors are pretty poppular among amateur and professional gamers. You can
see such tools exist from the late 90s and 00s where projects like
\texttt{GameMaker}, and \texttt{RPGMaker2000} surfaced, and up to today, in
2015, where \texttt{Unity} is ever emerging in popularity.

This is a relatively small and loose specification, where we elaborate on what
we want for the tool to do, and define a modular core we can build on, and let
users extend, via the finished product.

\subsection{About this Specification}

Originally I started writing this specification for fun. Slowly I piled up a few
design ideas, and the document took a fuller, and more complete form slowly, but
steadily. I want this specification to remain open, yet allow people the
opportunity to contribute, and improve it (and I'm sure this specification could
use a lot of input from more people). Therefore the document is licensed under
the creative commons, derivative works allowed, and attribution is required
(please link back to original if you want to fork the specification to a totally
different path --- else please make a pull request).
\\[0.2in]
\noindent The original location of the document is in the following link:

\url{\orionurllocation}

\subsection{Motivation (Userbase)}

A primary motivation for making this engine and game editor on the user side,
would be that many have the desire to work on a videogame but do not have the
resources. Another fact that would make this software package desirable is that
if there are requests for feature implementation in the engine, then since the
software is open source, they would be easilly implemented. An interested
userbase would help keep the project alive, even if previous developers decide
to abandon the software.

\subsection{Motivation (Developer)}

A lot of technical implications exist in game development, and in our case we
wish to provide the tool on many different platforms as well as a cross platform
engine. These aspects are all something that can contribute to the technical
growth of any contributing developer.

