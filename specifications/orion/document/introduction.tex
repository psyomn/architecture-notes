\section{Abstract}

The goal is to create a cross platfomr game maker, licensed under an open source
license, which is specifically targeting \texttt{JRPG} games.  The project has
two big parts: a flexible game engine, and a game editor. The game editor will
need to create, edit, and manipulate assets, and provide them in a correct form,
understandable by the engine. The engine should be flexible enough to adapt for
any game environment that is generated from the editor.

On the side, it would be nice to create a small and loose specification, or at
least a document where a lot of questions are asked and addressed, in order to
document problems for future implementations, or even different implementations
of such a game maker.

\section{Introduction}

Game editors have been poppular among amateur and professional gamers. You can
see such tools exist from the late 90s and 00s where projects like
\texttt{GameMaker} \cite{gamemaker}, and \texttt{RPGMaker2000}
\cite{rpgmaker2000} surfaced, and up to the time of writing in 2015, where
\texttt{Unity} is ever emerging in popularity, and some popular titles such as
\texttt{Rust} (\textit{Facepunch studios}), \texttt{Hearthstone}
(\textit{Blizzard}), \texttt{Mevius Final Fantasy} (\textit{Square Enix}), are
being developed by known game companies.

This is a relatively small and loose specification, where we elaborate on what
we want our custom designed game maker to do, and define a modular core we can
build on, and let users extend via the finished product. We wish to design and
implement a game making tool which will allow users to particularly make RPG
games easilly. In effect, we would not be as focused for other aspects in the
resulting games such as physics, or other features extensively elaborated upon
in commercial grade game makers. We strictly want a tool best for this genre,
and one that does a good job at that. In other words, we are not interested in
designing a tool where you could make any game --- our scope is narrowed down,
and well defined.

In this document, we describe a two part project, where each component should be
worked upon at the same time. The heart of the project lies in the engine of the
game editor. A game engine is required that will understand some predefined
setup, where it will look for assets, load and manipulate them to achieve
different game environments. In other words, we want a engine that will be
easilly extendable from the end-user's perspective.

The second part of the project, which again, must be developed parallel to the
engine, is the game editor. The game editor in effect is an asset management
and editing tool, which should be capable of generating some organization of
these assets such that the engine will be able to readilly take all the
information in, and load a different game environment. The game editor should
reuse parts of the core engine such that the environment in design, and in
execution is consistent. It also would be nonsensical to duplicate redundant
work.

This particular project will aim to be open source. At the time of the writing,
we wish to release under GPLv3. This does not strike out the possibility of
selling the end games generated by our tool, as the engine could be
redistributed without changes, but with the assets which will be different with
every game. This is still subject to change, and if we find that GPLv3 is too
restrictive with our vision, we may decide to adopt the \texttt{MIT} license.
Regardless, similar arrangements can be seen with the \texttt{Ren'py} engine
\cite{renpy}, and how some of games using this engine are being sold on Steam.

Finally, along the development of this game editor, the collaboration between
the developers, and the interested party aim to enhance this technical document
with documentation of unanticipated obstacles that are found along the way
whilst implementing the engine, and provided under a creative commons license.

\subsection{About this Specification}

Originally I started writing this specification for fun. Slowly I piled up a few
design ideas, and the document took a fuller, and more complete form slowly, but
steadily. I want this specification to remain open, yet allow people the
opportunity to contribute, and improve it (and I'm sure this specification could
use a lot of input from more people). Therefore the document is licensed under
the creative commons, derivative works allowed, and attribution is required
(please link back to original if you want to fork the specification to a totally
different path --- else please make a pull request).
\\[0.2in]
\noindent The original location of the document is in the following link:

\url{\orionurllocation}

\subsection{Targeted Audience}

A primary motivation for making this engine and game editor on the user side,
would be that many have the desire to work on a videogame but do not have the
resources. Another fact that would make this software package desirable is that
if there are requests for feature implementation in the engine, then since the
software is open source, they would be easilly implemented. An interested
userbase would help keep the project alive, even if previous developers decide
to abandon the software.

\subsection{As a Developer}

A lot of technical implications exist in game development, and in our case we
wish to provide the tool on many different platforms as well as a cross platform
engine. These aspects are all something that can contribute to the technical
growth of any contributing developer, and quite possibly rank as a good capstone
project. On the other hand, a good capstone project implies risk of failure, of
the whole project.

\subsection{Acknowledgements}

Many thanks to Thomas Giannakodimos, Felix Soumpholphakdy.
