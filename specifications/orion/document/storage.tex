\section{Storage}

In this section we describe possible strategies on how to persist the generated
assets by the game editor. One thing the reader needs to keep in mind is that
filesystems are not cross platform. The most obvious example would be that on
Windows machines, paths are delimited by a forward slash, and in Unix variants,
they are delimited by a backslash. A less obvious example would be filename
size. For example, there are different filename limits on different filesystems.
Hence producing different assets on one machine might not work on another
machine if you are transferring them as plain files. Or even different character
support on different filesystems: on ext4, you are able to use certain
characters and symbols, that NTFS might not allow. This is an aspect that the
game editor should take into account (and in effect the game engine as well).

Another thing to think about is if you would possibly want your engine to read
special zip files with more metadata about the game. For example, once the game
the user is designing, is complete, and is to be released, packaging it in a zip
format (or alike: example: cbr for comic books) for distribution, could be 
preferable. This is a nice to have feature, however.

\subsection{Proposed Directory Structure for Assets}

We can outline a possible directory structure that should be standard for any
given game that would be created by the editor. Ultimately this should take
care of any file organization that is required by the userbase. We should keep
in mind that allowing the user to create their own directories for other assets
should also be taken into consideration.

\begin{lstlisting}
  /items.xml
  /graphics/tilesets
  /graphics/portraits
  /sounds/music
  /sounds/sfx
  /sounds/speech/en
  /sounds/speech/el - Example, a Hellenic voice translation
  /lang/en
  /lang/el - Example, a Hellenic translation
\end{lstlisting}

