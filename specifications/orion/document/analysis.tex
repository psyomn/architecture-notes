\section{Analysis}

\subsection{Describing Abstract Data Types}

We will be using a form of algebraic specifications in order to describe the
data structures, in an abstract form. This is in the hopes to provide abstract
enough ideas, without imposing technical details too much. It will be up to the
implementers of the engine to find out the best way to implement the features
of the structures, in case there are possible discrepancies.

\subsubsection{Examples}

Consider the specification in \ref{eq:as:e1}. This is how we will represent
data structures in this document. This would translate to listing \ref{lst:as:e1}
in Java code. We also see some types later on in the specification, which are
essentially lists. We denote these attributes by surrounding them with angled
brackets. To see such an example, one may observe specification \ref{eq:as:e2},
where we define a structure, that has a label, and aggregates a list of numbers.
Since in specification \ref{eq:as:e2} there is a sequence, there would be many
data structures one could use in Java to implement this. This abstraction is
what is hoped to be achieved in this document (though at the time of
implementation it should be very well known on carefully thought decisions
what should be used).

\begin{equation}
\label{eq:as:e1}
\begin{split}
Name\colon String \\
Surname\colon String \\
Age\colon \mathbb{Z} \\
Person\colon Name \times Surname \times Age
\end{split}
\end{equation}

\begin{lstlisting}[caption=Representing the algebraic specification in Java
                  ,language=Java,label=lst:as:e1]
class Person {
  private String name;
  private String surname;
  private int age;
}
\end{lstlisting}

\begin{equation}
\label{eq:as:e2}
\begin{split}
Label\colon String \\
MyStructure\colon Label \times \langle \mathbb{Z} \rangle
\end{split}
\end{equation}

\subsection{Overall Description}

Ultimately we want two things. A game engine, and a game editor. The engine is
not to be modified - a standalone binary, or the like, must be provided, that
will look in predetermined locations for assets and tie the whole game
together.

Assets contain anything from text, graphics and sounds. Assets can be organized
in subcategories as well, since for example, in games we would like to discern
between background music, and sound effects. This organization is purely
aesthetic, and to help the user navigate through many things. Assets should
contain the following:

\begin{description}
  \itemit {Graphics}: Anything that has to do with things being drawn on screen.
    Tiles, Tilesets, portraits, backgrounds, etc. A good thing to do before
    proceeding to design and implementation details of this project at this
    point, is to decide what kind of graphic formats people would be interested
    in supporting in the engine. For instance, would \textit{JPEG} be favored?
    What about \textit{PNG}? Or is \textit{BMP} easier to manipulate? See what
    libraries exists, and choose something adequate.

  \itemit {Sound}: Anything that causes noise. Background music, sound effects, and
    anything else that falls into this category. Again, consider what sound
    formats you would like to support. Probably compressed formats are best such
    as \textit{Ogg Vorbis} or \textit{MP3}.

  \itemit {Text}: Anything that can be written down in a natural language. As a
    consequence, a mechanism should exist that will allow for multiple
    translations of the game, without invasive intervention (\textit{a.k.a.
    i18n, l10n}). Rails uses an organization, which involves YAML parsing
    \cite{ruby:i18n}, and Qt also does something similar \cite{qt:i18n}.

  \itemit {Map}: A part of the world the game takes place in. A map will contain
    information about what tilesets it's using, and the specific tiles used from
    that tileset. So in other words, each time a tile is set in the map, there
    should be a tuple stored somewhere, which contains an id of the tileset being
    used (or possibly a path), and an id which references the tile inside of that
    tileset.

  \itemit {Map Layers}: A map can have many layers, and transparency of images must
    be supported. There are two reasons: you might want to have layers that add
    extra niceties in the form of graphics. For example seeing things from an
    overhead view, and having leaves of trees (or branches) be over a player, yet
    see the player under the branches. Another motivation would be that we can add
    obstructions on different layers, meaning that the player could go on
    different floors of a smaller building for example, while everything is in
    fact on only one map. For example think of a bridge, where the player can move
    under, and over.

  \itemit {Tile}: a (usually) square graphic that can be used as a pattern to fill
    in the space of the map. For example a ``grass pattern'' can be expressed in
    one square graphic, and reused in order to build the map. Tiles exist on
    \textit{map layers}.

  \itemit {Tileset}: A collection of tiles (as described above). Tilesets usually
    have a particular style, or respect a scene (castle tilesets, should have
    things that are found in castles, overworld should have grass and trees, etc).


  \itemit {Pawn Templates}: When designing maps, there might be pawns you want to
    use, which are very common. There should be a mechanism to allow this. For
    example, the game editor should help the user out by providing a `Pawn
    Template Editor', which let's say, the user could define a sign-post, that
    the user can place in any town. As a result, the user would simply need to
    change the text of the sign.

  \itemit {Scripts}: Any behavior we want to define, without recompiling the
    engine. Apart from a sane design, this means that the user only has to worry
    about the scripting language, rather than tweak the game engine.

  \itemit {Entity}: An entity in a game, is anyting that might have some stats, and
    more information attached to it (usually these are monsters, can be nps, and
    the heroes the player can control).

  \itemit {Pawn (nonstandard terminology)}: A pawn is a placeable `thing' in a
    map, on a layer. These differ to tiles, in the sense that they are animated,
    and more than simple tiles on a map. For example NPCs, treasure chests, and
    anything else that might be deemed special. Pawns should contain some
    scripted behavior, that can tie the game together. For example it should be
    possible to perform map transitions via some minor scripting language, via a
    Pawn. These functionalities should include more things like changing the
    sprite, and more.

  \itemit {Portal}: A connecting unit between map `A' and map `B'. There might be
    many ways going about this --- maybe this is a special feature of a map, or
    there can be reusable functionality coded inside the engine, which would be
    used via a placed entity on the map, to perform this transition. A Portal, may
    be a Pawn.

  \itemit {Sprite and Spritesheets}: a sprite is similar to tiles. The term is
    used for graphics, but in this case, the difference is that we associate
    animation with this term. A landscape will use tiles and tilesets for a
    graphical representation. The pawns, or other things that might be moving on
    the map, or are interacted with, will be using sprites. Sprites use the same
    setup as tiles --- a grid that separates graphics. The difference is that
    each `tile' in the spritesheet is the next animated frame.

\end{description}

Another aspect that we need to think along the way is how modular we want
the game engine to be. We need a balance between how much the engine provides
to the userbase, and how much the users need to work in order to get the game
as they want it. For example, some would be satisfied with a message box that
provides the dialog with just plain text, and some might prefer a message box
that also has a portrait on the side. Additionally others would want the
portraits to vary in emotional expressions. These are all very specialized
aspects, and pushes us towards a more stripped down product, but modular to
allow for these maneouvers.

Due to the extreme modular requirements of such a software product, it might be
wiser to use a scripted programming language along the game engine. The rationale
being that we provide some modular engine, which is possibly somewhat stripped
down, but allow the users of the game editor to specify the missing parts, with
the engine's scripting language. For example, we impose the the requirement on
the developers to create a modular engine, that is able to understand some
scripting language; we omit menus in the game engine, but through the scripting
language, we would be able to provide it to the userbase. On the other hand,
trying to make the engine modular to this extent might not be feasable. It is up
to the developers to make this decision.

\subsection{Methods of Interaction, Controller Input}

Ultimately the editor and game engine should be able to be modular enough to
specify different keys for different interactions. We do not want to hardcode
for example an ``action button'', a ``menu button'', and so forth. However we
can take into consideration the traditional \texttt{NES} controller, and how
such a minimalistic set of buttons have been able to provide most basic
functionalities for an RPG (and to this day a very similar format, if not
entirely exact is still being followed for the layout of said action buttons).
The developers might want to consider a standard layout the game engine might
start off with, and let the users extend it as they see fit. The engine
\textit{should support keyboard primarily}, and provide joystick support as
well. In the case that the joystick doesn't work, fallback to keyboard.

Another aspect to think about here is external joystick support. Would it be
possible to target a brand of joysticks that support a variety of operating
systems? A good candidate to look at for early system testing would be
\texttt{Logitech} controllers, as I personally have never had any problems to
get them to work on any system. The developers should think about adapters in
this case --- support a \texttt{Logitech} controller first, and see what the
interface is like. Make some adapters (I'm suggesting software patterns), which
will provide extensibility in the future.

\subsection{Graphics}

In the respect of graphics, we want Orion to be a retro game engine. So in this
respect we will be interested mostly with the manipulation of 2d graphic files.
However blending in 3d aspects to 2d environments is totally possible and could
prove to be desirable by the userbase in the future.

For the graphics we require a library that can handle image manipulation. The
most important aspect being croping as we want to extract different tiles from
the tilesets that will exist in the assets.

More trivially this part of the software should be able to load bigger pictures
such as portraits and backgrounds for the games as well.

\subsection{Text}

Text is more important in role playing games as they traditionally are games
with richer storylines. Therefore it is even greater motivation to detect this
and prepare a proper mechanism.

A mechanism that has worked in the past in several applications I had to work
with and proved to be the right path to take in development, is
\textit{internationalization} and \textit{localization} of the system. Those
are big words for simply meaning, storing the text of the application separately
and telling the system to look for an \textit{id} that containts the wanted
text in that moment.

A simple example would involve a menu asking your name. There would be a label
with the caption ``name''. For English speakers, that would be fine. But what
if someone wishes to use the application who does not know how to communicate
in that language? Should we make a copy of the project, and replace all the
string instances with ``name'' and others to that other language? Or should we
factor out this functionality in some other way?

That is exactly where the said mechanisms come in, and we can have multiple
translations for the same application available, ready to update, and even add
more languages without the need of reloading or recompiling the application. In
the future if let's say a game is made by this editor and engine, and it is
accepted, we would like to provide a means for translators to have no trouble to
go about their work.

\subsection{Sound}

Sound is pretty much anything that makes noise while the player is playing the
game. If more recent technology is to be used in order to write the music, then
I would suggest the use of Ogg Vorbis and MP3. The engine is open source and Ogg
Vorbis is also open source. Some distributions, or even users tend to be
restrictive on what codecs they want to have, and sometimes they blacklist the
MP3 format as well.

For background music, we want to be able to loop the song. It would be also
quite advantageous to point exactly where we want it to loop, as a song might
have an intro that is not repeated later on. I've seen this being omitted a few
many times and it's something that I personally did not like when I found it
lacking.

Sound effects should be possible to mess around with effects. For example
switching the pitch slightly for a repeating sound might make it sound less
monotonous and not irritate the user as well.

Also, another thing to take into consideration is voice actors, and games that
not only have the text of speech displayed, but actually can enabled it to have
the speech spoken out. This is another aspect of internationalization and
localization that we need to take care off later on as well. The text should
have more metadata bound, such that referencing these voice samples is possible.

\subsubsection{Tiles, and Tilesets}

The maps should be illustrated with tilesets. The tilesets should be on the
same dialog of the map editor, as should be the aforementioned layers. A map
can be designed using many different tilesets. A demonstration of the
suggested layout to provide a comprehensive and usable interface to the user
is shown in Figure~\ref{ui:map-editor}. In the specification
\ref{eq:tile-explain}, we notice that \texttt{Tile} is composed of two things:
tileset ids, and offsets. What this means is that we know that the tile will
look for some tileset of known origin (thanks to some identifier), and will be
given an offset. The offset in this context can be a natural, including 0
($\mathbb{N}_0$) --- in other words $\{ 0, 1, 2, 3, ... \}$. These would label
specific tiles on the tileset. The tileset needs to specify what kind of grid
it has as well. And if the grid has thick borders. Then, from the grid we would
extract the tiles from left to right, and top to bottom. We count each tile in
the grid, with an offset number.

\begin{equation}
  \begin{split} \label{eq:tile-explain}
  GridWidth\colon \mathbb{N}_0 \\
  Tile\colon TilesetId \times TileOffset \\
  Tileset\colon ImageResource \times GridWidth
\end{split}
\end{equation}

\psyfig{figures/ui/map-editor.png}%
       {Map Editor UI}%
       {ui:map-editor}%
       {5in}

Now for a visual representation of the above specification, we can observe
Figure~\ref{sprites:arachne:1} which shows some sprite-art from a user called
\texttt{Arachne}, from \texttt{tigsource.com}.

\psyfig{figures/sprites/dungeontiles-test.png}%
       {Dungeon Tiles of Arachne on tigsource.com \cite{arachnesprite1}}%
       {sprites:arachne:1}%
       {3in}

As you may notice, just like real tiles, the tiles in
Figure~\ref{sprites:arachne:1} are of square shape; same length of sides. There
are other variations of tiles, such as hexagonal tiles seen in other games. For
this specification, we will focus on simple, square tiles. There is one
consideration to think of however: in some cases, some spritesheets or tilesets
contain a grid, who's lines have a width. Usually it's a size of 1 pixel, but
there should be mechanisms to deal with a variable length of pixels. We can see
such a thing in Figure~\ref{sprites:arachne:gridwidth}.

\psyfig{figures/sprites/dungeontiles-with-gridwidth.png}%
       {Dungeon Tiles of Arachne modified with grid-width}%
       {sprites:arachne:gridwidth}%
       {3in}

\subsection{Maps}

Maps are bits and pieces that build the world of the game. They are pretty
essential, and the correct analysis, design and implementation of them, in my
opinion, is critical. There's a lot of ways the maps can be expressed in a
system, but many of these ways can be considered wrong or lacking.

When we boil it down to basics, there's two aspects we really need to look out
for. One of them are moving entities, and the others stills. A quality one has
to deal with is also collisions, and appropriate actions to take on these
collisions. Role playing games are very simple in this aspect (but we should
still provide a rich set of actions that may be performed for the sake of
expressiveness of our users). In specification \ref{eq:bigmapspec}, we define a
few things the reader has not encountered yet. We recommend going over this
section a few times, as some of these properties are described more in depth in
their respective sections below.

\begin{equation}
\begin{split} \label{eq:bigmapspec}
Label\colon String \\
Metadata\colon \langle String \times String \rangle \\
TileMatrix\colon \langle \langle Tile \rangle \rangle \\
Tile\colon TilesetId \times TileOffset
\end{split}
\end{equation}

\begin{equation}
Layer\colon Label \times TileMatrix \times Metadata
\end{equation}

\begin{equation}
Map\colon ResourceId \times Label \times \langle Layer \rangle \times \langle Entity \rangle
\end{equation}

\subsubsection{Movement in Maps}

The reader should recall, or take a look at two games. One game is
\textit{Chrono Trigger}, and the other is \textit{Final Fantasy VI}. These two
games have similar if not same maps. The difference is how the players are able
to move on the maps. In \textit{Chrono Trigger} you are able to move in any
direction you please. In \textit{Final Fantasy VI}, you are constrained at the
up, down, left, and right directions, and these positions are fixed --- you only
move one tile at a time, and you can think of it as if the player were `snapped
to a grid'.

\subsubsection{Layers}

Layers are the bread and butter in RPG maps. We want to provide rich maps in
order for the user to explore. To provide these rich maps in the respect of
graphics, we rely on transparency for pictures, and a layering system. Each
map can have a set of layers that describe the current location in pieces so
that if a character were to be behind a tile with a hole, only the visible part
of the player would be visible. The same technique can be used for other things
such as graphics. For example if we were to have two maps with the same ``tree
graphics'', but they had different backgrounds, we would not want to create two
different trees for this. That would waste space, and create extra work. Hence
in the tree graphics, we would make their background transparent in order to
make them reusable with any background.

\begin{equation}
\begin{split}
Tile\colon TilesetId \times TileOffset \\
TileMatrix\colon \langle \langle Tile \rangle \rangle
\end{split}
\end{equation}

\subsubsection{Portals}

Portals connect maps together and make it possible for the player to travel
between points. Portals could be placed as if they were NPCs in order to make
the player move between maps, but should be implemented more flexible. For
instance it should be a reusable function so that the player can either step
on a tile to transport, or activate in a conversation with a NPC which would
teleport you instead.

\subsection{State Machines}

There are different events that may happen in the game. To keep track of
things, the most headache-free way would be to have a state machine mechanism,
preferably generic and reusable, ready to be manipulated by our engine. For
example game events could be processed this way. In order for some event to
happen in the game, we'd want to have some other event fired up. The easiest
example is speech which may be affected by in game events.

Having a state machine, would also centralize potentially global information
that you might not wish to have scattered around in different game scripts (for
example if the user was designing a game, and the engine supported a gaming
script, and the user defined a bunch of global variables in order to deal with
things changing, and reacting to those changes). It is also possible to visualize
state machines with tools such as graphviz as well.
