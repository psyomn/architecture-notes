\section{Analysis}

Ultimately we want two things. A game engine, and a game editor. The engine is
not to be modified - a standalone binary, or the like, must be provided, that
will look in predetermined locations for assets and tie the whole game
together.

Assets contain anything from text, graphics and sounds. Assets can be organized
in subcategories as well, since for example, in games we would like to discern
between background music, and sound effects. This organization is purely
aesthetic, and to help the user navigate through many things. Assets should
contain the following:

\begin{description}
  \itembf{Graphics}: Anything that has to do with things being drawn on screen.
    Tiles, Tilesets, portraits, backgrounds, etc. A good thing to do before
    proceeding to design and implementation details of this project at this
    point, is to decide what kind of graphic formats people would be interested
    in supporting in the engine. For instance, would \textit{JPEG} be favored?
    What about \textit{PNG}? Or is \textit{BMP} easier to manipulate? See what
    libraries exists, and choose something adequate.

  \itembf{Sound}: Anything that causes noise. Background music, sound effects, and
    anything else that falls into this category. Again, consider what sound
    formats you would like to support. Probably compressed formats are best such
    as \textit{Ogg Vorbis} or \textit{MP3}.

  \itembf{Text}: Anything that can be written down in a natural language. As a
    consequence, a mechanism should exist that will allow for multiple
    translations of the game, without invasive intervention (\textit{a.k.a.
    i18n, l10n}). Rails uses an organization, which involves YAML parsing
    \cite{ruby:i18n}, and Qt also does something similar \cite{qt:i18n}.

  \itembf{Maps}: A composite that brings many of the aforementioned assets
    together, to make up the virtual world.

  \itembf{Pawn Templates}: When designing maps, there might be pawns you want to
    use, which are very common. There should be a mechanism to allow this. For
    example, the game editor should help the user out by providing a `Pawn
    Template Editor', which let's say, the user could define a sign-post, that
    the user can place in any town. As a result, the user would simply need to
    change the text of the sign.

  \itembf{Scripts}: Any behavior we want to define, without recompiling the
    engine. Apart from a sane design, this means that the user only has to worry
    about the scripting language, rather than tweak the game engine.
\end{description}

Another aspect that we need to think along the way is how much modular we want
the game engine to be. We need a balance between how much the engine provides
to the userbase, and how much the users need to work in order to get the game
as they want it. For example, some would be satisfied with a message box that
provides the dialog with just plain text, and some might prefer a message box
that also has a portrait on the side. Additionally others would want the
portraits to vary in emotional expressions. These are all very specialized
aspects, and pushes us towards a more stripped down product, but modular to
allow for these maneouvers.

Due to the extreme modular requirements of such a software product, it might be
wiser to use a scripted programming language along the game engine. The rationale
being that we provide some modular engine, which is possibly somewhat stripped
down, but allow the users of the game editor to specify the missing parts, with
the engine's scripting language. For example, we impose the the requirement on
the developers to create a modular engine, that is able to understand some
scripting language; we omit menus in the game engine, but through the scripting
language, we would be able to provide it to the userbase. On the other hand,
trying to make the engine modular to this extent might not be feasable. It is up
to the developers to make this decision.

\subsection{Methods of Interaction, Controller Input}

Ultimately the editor and game engine should be able to be modular enough to
specify different keys for different interactions. We do not want to hardcode
for example an ``action button'', a ``menu button'', and so forth. However we
can take into consideration the traditional \texttt{NES} controller, and how
such a minimalistic set of buttons have been able to provide most basic
functionalities for an RPG (and to this day a very similar format, if not
entirely exact is still being followed for the layout of said action buttons).
The developers might want to consider a standard layout the game engine might
start off with, and let the users extend it as they see fit. The engine
\textit{should support keyboard primarily}, and provide joystick support as
well. In the case that the joystick doesn't work, fallback to keyboard.

Another aspect to think about here is external joystick support. Would it be
possible to target a brand of joysticks that support a variety of operating
systems? A good candidate to look at for early system testing would be
\texttt{Logitech} controllers, as I personally have never had any problems to
get them to work on any system. The developers should think about adapters in
this case --- support a \texttt{Logitech} controller first, and see what the
interface is like. Make some adapters (I'm suggesting software patterns), which
will provide extensibility in the future.

\subsection{Graphics}

In the respect of graphics, we want Orion to be a retro game engine. So in this
respect we will be interested mostly with the manipulation of 2d graphic files.
However blending in 3d aspects to 2d environments is totally possible and could
prove to be desirable by the userbase in the future.

For the graphics we require a library that can handle image manipulation. The
most important aspect being croping as we want to extract different tiles from
the tilesets that will exist in the assets. Just a simple reminder on some
jargon:

\subsubsection{Tileset}
Is an image file that contains smaller images that can be used repeatedly as a
pattern.

\subsubsection{Tile}
Is a smaller graphic that is a pattern, that can be used to portray repeated
graphics.

More trivially this part of the software should be able to load bigger pictures
such as portraits and backgrounds for the games.

\subsection{Text}

Text is more important in role playing games as they traditionally are games
with richer storylines. Therefore it is even greater motivation to detect this
and prepare a proper mechanism.

A mechanism that has worked in the past in several applications I had to work
with and proved to be the right path to take in development, is
\textit{internationalization} and \textit{localization} of the system. Those
are big words for simply meaning, storing the text of the application separately
and telling the system to look for an \textit{id} that containts the wanted
text in that moment.

A simple example would involve a menu asking your name. There would be a label
with the caption ``name''. For English speakers, that would be fine. But what
if someone wishes to use the application who does not know how to communicate
in that language? Should we make a copy of the project, and replace all the
string instances with ``name'' and others to that other language? Or should we
factor out this functionality in some other way?

That is exactly where the said mechanisms come in, and we can have multiple
translations for the same application available, ready to update, and even add
more languages without the need of reloading the application. In the future if
let's say a game is made by this editor and engine, and it is accepted, we
would like to provide a means for translators to have no trouble to go about
their work.

\subsection{Sound}

Sound is pretty much anything that makes noise while the player is playing the
game. If more recent technology is to be used in order to write the music, then
I would suggest the use of Ogg Vorbis. The engine is open source and Ogg Vorbis
is pretty awesome when it comes to application.

For background music, we want to be able to loop the song. It would be also
quite adventageous to point exactly where we want it to loop, as a song might
have an intro that is not repeated later on. I've seen this being omitted a few
many times and it's something that I personally did not like when I found it
lacking.

Sound effects should be possible to mess around with effects. For example
switching the pitch slightly for a repeating sound might make it sound less
monotonous and not irritate the user as well.

Also, another thing to take into consideration is voice actors, and games that
not only have the text of speech displayed, but actually can enabled it to have
the speech spoken out. This is another aspect of internationalization and
localization that we need to take care off later on as well.

\subsection{Maps}

Maps are bits and pieces that build the world of the game. They are pretty
essential, and the correct analysis, design and implementation of them, in my
opinion, is critical. There's a lot of ways the maps can be expressed in a
system, but many of these ways can be considered wrong or lacking.

When we boil it down to basics, there's two aspects we really need to look out
for. One of them are moving entities, and the others stills. A quality one has
to deal with is also collisions, and appropriate actions to take on these
collisions. Role playing games are very simple in this aspect (but we should
still provide a rich set of actions that may be performed for the sake of
expressiveness of our users).

\subsubsection{Layers}

Layers are the bread and butter in RPG maps. We want to provide rich maps in
order for the user to explore. To provide these rich maps in the respect of
graphics, we rely on transparency for pictures, and a layering system. Each
map can have a set of layers that describe the current location in pieces so
that if a character were to be behind a tile with a hole, only the visible part
of the player would be visible. The same technique can be used for other things
such as graphics. For example if we were to have two maps with the same ``tree
graphics'', but they had different backgrounds, we would not want to create two
different trees for this. That would waste space, and create extra work. Hence
in the tree graphics, we would make their background transparent in order to
make them reusable with any background.

\subsubsection{Portals}

Portals connect maps together and make it possible for the player to travel
between points. Portals could be placed as if they were NPCs in order to make
the player move between maps, but should be implemented more flexible. For
instance it should be a reusable function so that the player can either step
on a tile to transport, or activate in a conversation with a NPC which would
teleport you instead.

\subsubsection{Tiling}

The maps should be illustrated with tilesets. The tilesets should be on the
same dialog of the map editor, as should be the aforementioned layers. A map
can be designed using many different tilesets. A good demonstration of the
required layout to provide a comprehensive and usable interfacr to the user
is laid out in Figure \ref{ui:map-editor}


\psyfig{figures/ui/map-editor.png}%
       {Map Editor UI}%
       {ui:map-editor}%
       {5in}


