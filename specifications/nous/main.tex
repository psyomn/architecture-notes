\documentclass[12pt,twoside]{article}

\newcommand{\projectname}[0]{\textit{nous}}

\title{\projectname: A simple monitoring system}
\author{Simon Symeonidis}
\date{Fri Feb  2 11:01:51 EST 2018}
\usepackage[margin=1in]{geometry}
\usepackage{listings}
\usepackage{color}
\usepackage[colorlinks]{hyperref}
\usepackage{graphicx}
\usepackage{grffile} % added to understand *.png
\usepackage[export]{adjustbox}
\usepackage[labelfont=bf]{caption}
\DeclareCaptionFont{white}{\color{white}}
\DeclareCaptionFormat{listing}{\colorbox{psyan}{\parbox{\textwidth}{#1#2#3}}}
\captionsetup[lstlisting]{format=listing,labelfont={white,bf},textfont={white}}

%%% Macros
%%%
%%% Macro shorthand to add figures
%%%

\newcommand{\psyfig}[3]{%
    \begin{figure}[h!]
    \centering
    \includegraphics[max width=\linewidth, keepaspectratio]{#1}
    \caption{#2}
    \label{fig:#3}
    \end{figure}}

    %% TODO LEGACY: need to fix other document parts using this
\newcommand{\psyfigd}[3]{%
    \begin{figure}[h!]
    \centering
    \includegraphics[max width=\linewidth, keepaspectratio]{#1}
    \caption{#2}
    \label{fig:#3}
    \end{figure}}


%%%
%%% Rust Latex Macros
%%%

%%
%% Add posibility to boldface first word of item
%%

\newcommand{\itembf}[1]{%
  \item \textbf{#1}
  }


\pagenumbering{arabic}

\begin{document}

\maketitle
\tableofcontents
\listoffigures
\lstlistoflistings

%%% Not actually a macro - forced to declare here due to lstset et al.
\definecolor{blue}{RGB}{0,0,255}
\definecolor{green}{RGB}{0,100,0}
\definecolor{psyan}{RGB}{0,104,222}
\definecolor{lightpsyan}{RGB}{185,239,254}
\definecolor{red}{RGB}{255,0,0}

\lstdefinelanguage{nouscontract}{
morekeywords={
FROM,PORT,GET,ENSURE,WITH,>,<,=,
PERSIST,OPTIMISTIC,PESSIMISTIC,NONE},
otherkeywords={},
sensitive=true,
morecomment=[l]{//},
morecomment=[n]{/*}{*/},
morecomment=[l]{--},
morestring=[b]",
morestring=[b]',
morestring=[b]"""
}

\lstset{ %
  basicstyle=\ttfamily\small,        % the size of the fonts that are used for the code
  breakatwhitespace=false,         % sets if automatic breaks should only happen at whitespace
  breaklines=true,                 % sets automatic line breaking
  captionpos=t,                    % sets the caption-position to bottom
  commentstyle=\color{green},      % comment style
  deletekeywords={...},            % if you want to delete keywords from the given language
  escapeinside={\%*}{*)},          % if you want to add LaTeX within your code
  extendedchars=true,              % lets you use non-ASCII characters; for 8-bits encodings only, does not work with UTF-8
  frame=b,                    % adds a frame around the code
  keepspaces=true,                 % keeps spaces in text, useful for keeping indentation of code (possibly needs columns=flexible)
  keywordstyle=\color{psyan}, % keyword style
  language=nouscontract,                      % the language of the code
  morekeywords={*,...},            % if you want to add more keywords to the set
  % numbers=left,                    % where to put the line-numbers; possible values are (none, left, right)
  % numbersep=5pt,                   % how far the line-numbers are from the code
  % numberstyle=\tiny\color{lightpsyan}, % the style that is used for the line-numbers
  rulecolor=\color{black},         % if not set, the frame-color may be changed on line-breaks within not-black text (e.g. comments (green here))
  showspaces=false,                % show spaces everywhere adding particular underscores; it overrides 'showstringspaces'
  showstringspaces=false,          % underline spaces within strings only
  showtabs=false,                  % show tabs within strings adding particular underscores
  stepnumber=2,                    % the step between two line-numbers. If it's 1, each line will be numbered
  stringstyle=\color{red},         % string literal style
  tabsize=2,                       % sets default tabsize to 2 spaces
  framexbottommargin=5pt,
  framextopmargin=5pt,
  title=\lstname                   % show the filename of files included with \lstinputlisting; also try caption instead of title
}


\section{Summary}
Ensuring that large systems are working well together can be
very messy. As a result simple problems that could be avoided
get neglected and build up to problems that can result in
failures.

We treat systems as black boxes, and view them connected by
the data flow that enters and exits each of them. These systems as
black boxes are then monitored with a publish or subscribe method to a
central repository, that stores and alerts about infractions on
defined contracts.

The contracts are defined with a simple DSL, which declare certain
endpoints of systems, where they can be queried about their
status. For the implementation of \projectname this is done via
\textit{JSON REST} endpoints, being queried and compared against each
other.

\section{Observed System States}
There can be many states the system is in. For \projectname we define
states `OK', and `BAD' to be the only two possible states a system can
be in. An intermediate state makes no sense.

\subsection{Possible States}
\begin{itemize}
\item OK
\item BAD
\end{itemize}

\subsection{When Monitored Service has Crashed}
In case the service we are monitoring has crashed, sending a request
and getting a response is impossible. We care about the following:

\begin{itemize}
  \itembf{Retries}: Logarithmic backoff. Retry a few times until the
    state of the system is retrievable.
  \itembf{Auto-Bad}: Until the state of the system is retrieved, we
    assume that the status of the monitored system is `BAD'.
\end{itemize}

\section{System REST endpoint}
REST endpoints must be adhere to a very simple JSON format in order to
be queriable by \projectname.
\begin{lstlisting}[caption='Sample REST endpoint]
  {
    "status": "OK",
    ...
  }
\end{lstlisting}

\subsection{What about loose variables without state?}
There may be cases that we want to expose information that makes state
redundant. For example if we were to just expose a \textit{semver} of
the monitored running system, it would not make sense to attach a
\textit{status} environment variable.

\paragraph{Problem}: The specification requires that all maps, have at
their root a status element, however this doesn't make sense with the
above (for example exposing a \textit{semver}.

\section{Contracts}
\subsection{Contracts as DSLs}
Should be human readable SLAs. Human readable documentation: enforce
documetation Conditions on served json from different endpoint. This
is done with the `WITH' keyword, which is not optional. We see a
DSL/Sketch of this contracts as such:

\begin{lstlisting}[language=nouscontract,caption='Sample Contract']
-- NOTE: all of the endpoints should be returning JSON. We are
-- assuming JSON everywhere, though there could middleware that deals
-- with other formats.
--
-- TODO: do we care about invariants?
-- Collect different variables from different endpoints that
-- eventually will be needed to form the contract
--
-- TODO: what kind of access do we want for the keys?
-- TODO: what about array index access?
FROM www.a.com/status PORT 1234
 GET .key1 AS a_key1,
     .key2 AS a_key2,
     .keyx.innerkey AS a_innerkey;

FROM www.b.com/otherplace/status
PORT 80
 GET .key1 AS b_key1,
     .key2 AS b_key2,
     .keyx.innerkey AS b_innerkey;

FROM 123.32.21.23/rofl/hahaha/status&derp=json
PORT 9001
 GET .key1 AS c_key1,
     .key2 AS c_key2,
     .keyx.innerkey AS c_innerkey;

ENSURE                  -- example contract
  a_key1 =  b_key2 AND  -- logically this means nothing, just
  b_key1 <  a_key2 OR   -- sketching to see what I care about
  c_key1 >  a_key3 AND  -- dsl-wise
  c_key1 >= a_key3 AND  -- ie: want and, or logical ops, and
  c_key1 <= a_key2 OR   -- <, >, <=,  >=, <> (not)
  c_key3 <> b_key3

WITH
  '
  This is some text that will go here and should go here
  and support multiline stuff. Also it should describe what
  went wrong. For example, this contract would say that
  "a_key1 must be equal to b_key2, because we have some
  client that required the system a certain way."
  ';

PERSIST OPTIMISTIC;
\end{lstlisting}

\subsection{Contracts piggybacking on Language Features}
On the other hand, contracts don't need to necessarilly be a DSL. For
example it would be interesting to see contracts being dynamically
loaded with a language like Erlang. An advantage to dynamic loading is
that the contracts do not trigger a complete recompile of the system,
allowing them to be tweaked during runtime.

\subsection{Fast feedback on contract loading}
\projectname \ plans to allow for quick and easy ways to load contracts,
such that tweaking them in runtime is something pleasant for the
programmers using this service. As such we need to worry about things
getting loaded, or erroring out before actually being loaded, and
giving direct feedback.

\subsubsection{Notification through logs}
Tailing logs tend to be pretty standard for this sort of thing. This
requires an extra terminal, and running a command on the side.

\subsubsection{Notification through filesystem}
A simple way to poke users, to show if things loaded or not is to do
it through empty files on the contract folder. For example if we had a
setup like the following:

\begin{verbatim}
contracts/
  c1.contract
  c2.contract
\end{verbatim}

We could create files appended with `loaded' or `error', to denote
what went well or wrong. This way we don't need to tail the logs
(though logs may be idependantly be produced). For example if
\textit{c1} was loaded and \textit{c2} had problems being loaded we
could get the following:

\begin{verbatim}
contracts/
  c1.contract
  c1.contract.loaded
  c2.contract
  c2.contract.error
\end{verbatim}

Cleanups can probably be a pain in this case, and I foresee that it
might add a lot of crappy code this way, to manage this feature.

\subsection{Expiry of Contracts}
It may be the case that certain contrats need expiry or re-evaluation
after some time. A sanity poke here would be nice. In other words, the
ability to fail a contract regardless of criteria, after a certain
date.

\section{Layout of Platform}
Registry like platform as it connects a bunch of services
together. Subscribe architecture is chosen in this case, as it can be
the case that certain services may never publish their state if things
go wrong -- in other words, it's better for \projectname to query, and
deduce problems of services.

\section{Security}
For this particular system, we provide no authentication
mechanism. This concern should be pushed to the firewalls of the
machine. It would not make sense to publicly expose this service
anyway.

\section{History}
Should we worry about storing snapshots of things over time? By
doing this, we could have a easy way for querying things.

\paragraph{Preferred: Optimistic Storing}
Store everything. This, though wasteful, can give a better view of
what is happening to the system over a span of time.

\paragraph{Considered: Pessimistic Storing}
Store only infractions. This will store less things but on the
other hand will not allow for an overall view as the above. This is
not great if the system is to be able to monitor for anomalies in the
future. However, this can be enabled by certain flags when the system
is running, and left up to the user of the system to define when this
will benecessary.

\subsection{Storage Method}
Snapshots of infractions (probably json?). Don't need complicated
store. Maybe even Mnesia would be good for this use case.

\section{Runtime Requirements}
Being able to dynamically ship contracts. This could be done
through networked file transfer for example. ie: have a fodler
`contracts/`, that gets poppulated with contracts shipped by some
technique.

\subsection{Reliability}
What hard promises can this service provide?
What happens if this service crashes?

\section{Alerting Mechanisms}
Methods of nofifications for when something goes bad.
\begin{itemize}
  \itembf{Messaging/Poking Module}: It would be nice to send out
    alerts in some maner. Due to the many available services, this
    part could be logically implemented as a plugin.
  \begin{itemize}
    \itembf{Send an email}: (and eventually maybe even fine grain to
      poke people responsible for failures).
    \itembf{Ping person/people in chat}: Poking someone on IRC or some
      other chat client is also nice.
  \end{itemize}
\end{itemize}

\section{Questions and Notes}
\begin{itemize}
  \itembf{Service Disovery}: This might be a little problematic: how can
    we know where services with different statuses exist?

  \itembf{YECC}: (erlang parser generator) to make parser that
    understands contract dsl.

  \itembf{Contracts Backwards Compatible}: It should be obvious that
    contracts that you specify in one point in time should be still
    usable in the future.

  \itembf{Testing Strategy}: Since a lot of things rely on services
    built around \projectname we'll have to think about how to
    effectively write tests that make sense.

  \itembf{Continuous Integration}: Would be good to be able to query
    questionable services like Travis CI, to monitor build time of
    projects, and how they're faring over time (ie build times
    dramatically increasing between two builds). Susceptible to frog
    boil.

\end{itemize}
\end{document}
