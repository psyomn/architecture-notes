\section{Core Engine Features}

This section describes a few less technical aspects of the game engine. For
example, the Entities section (section \ref{sec:entities}), describes of a
modular organization of a small stats engine, that may be defined dynamically by
the user.

\subsection{Entities}
\label{sec:entities}

We will now discuss about a possible organization for the data that an entity
mahy hold and manipulate. Essentially an entity should be composed of `hard
facts', and `modular facts'. Hard facts of an entity would be things such as the
entitie's \textit{strength}, \textit{magic}, and other similar information. Such
statistical information should be represented as real numbers: $\mathbb{R}$.
When it is time for implementation, you might wish to prefer \texttt{float}
types instead of \texttt{double}, as the extended precision of the latter is not
required. Each of these attributes, come bundled with a label, which will be
used to tell the player what kind of attribute the player is observing. We can
therefore express this as seen in the algebraic specifications
\ref{eq:attribute}, and \ref{eq:entity}.

\begin{equation}
\begin{split}
  \label{eq:attribute}
  Label\colon String \\
  Attribute\colon Label \times \mathbb{R}
\end{split}
\end{equation}

Now let us describe the module-plugin organization. Since we want to exhibit
different behaviors in the stats of an Entity, we want the Entity to maintain
a list of these behaviours. Notice that these behaviors take in a type
\texttt{Entity}, and map it to an \texttt{Entity} as well. So these behaviors
should essentially alter the state of the \texttt{Entity}, when fired up. So for
instance, you could create different functions for a \textit{Fighter}, and
define the attributes affected on a level up. This does not exclude the
possibility of adding more functions, and composing a function out of different
behaviours. For example, if the player is cursed, there could be a function that
gives a penalty per level up. We show such an organization in
Specifications~\ref{eq:funcmodules}, and later how it is bound to the entity in
\ref{eq:entity}.

\begin{equation}
\begin{split}
  \label{eq:funcmodules}
  f \colon Entity \to Entity \\
  MODULES_{seq}\colon \langle f_1, f_2, ..., f_n \rangle
\end{split}
\end{equation}

Another essential aspect to RPG's is the equipment system, which allows the
player to tweak the performance of the \texttt{Entity}. This is simply a type,
which holds a label, and any number of \texttt{Attribute}s. This way we can
specify modifiers for things like strength, intelligence, and more, as well as
add other things like elemental damage. We can see this in specification
\ref{eq:equipment}.

\begin{equation}
\begin{split}
  \label{eq:equipment}
  Equipment\colon Label \times \langle Attribute \rangle
\end{split}
\end{equation}

Now, for the sake of flexibility, we should provide a set of functions that will
perform checks each time a piece of equipment is attempted to be equiped for the
character. This should take in an \texttt{Entity}, an \texttt{Equipment} and
yield a \texttt{Boolean}, as demonstrated in
specification~\ref{eq:equipmentcheck}. This way, we define a sequence which
contains all these constraints as well, with the name
\texttt{EquipmentConstraints}. For setups where each character can equip
anything, then this may be set as an empty list.

\begin{equation}
\begin{split}
  \label{eq:equipmentcheck}
  f\colon Entity \times Equipment \to Boolean \\
  EquipConstraints_{seq} = \langle f_1, f_2, ..., f_n \rangle
\end{split}
\end{equation}

We should also specify some container in the implementation, to contain one item
from the enumeration shown in specification~\ref{eq:entityclass}. An interesting
question is if it would be possible to mix any number of classes, to get a mixed
class, and in turn automatically constrain skills or buffs on an entity. For now
we stick to single class assignment as seen in
specification~\ref{eq:entityclass}. And finally we can define our composite type
\texttt{Entity} with the specification shown in \ref{eq:entity}.

\begin{equation}
  \label{eq:entityclass}
  EntityClass = \{ Fighter, Mage, Thief, Assassin, Archer, ... \} \\
\end{equation}

\begin{equation}
\begin{split}
  \label{eq:entity}
  Entity\colon Label \times
    EntityClass \times
    \langle Attribute \rangle \times
    \langle Equipment \rangle \\
    \times MODULE_{seq} \times EquipmentConstraints_{seq}
\end{split}
\end{equation}

Note that since $MODULE_{seq}$, and $EquipmentConstraints_{seq}$ are both
sequences of functions, with same type, which implies that we can compose them,
in order to get a wanted result. So for instance, the final form of the entity,
could be represented as shown in equation~\ref{eq:entitycomposition}. In
equation \ref{eq:equipconstrcomp} we can see a similar approach, to check that
any constraints that the entity may have, are satisfied or not, yielding a
boolean value in the end.

\begin{equation}
\begin{split}
  \label{eq:entitycomposition}
  Entity_{final} = (f_1 \circ f_2 \circ ... \circ f_n)(E_{simple}) \\
  where, \langle f_1, f_2, ..., f_n \rangle \in MODULE_{seq}
\end{split}
\end{equation}

\begin{equation}
\begin{split}
  \label{eq:equipconstrcomp}
  CanEquip = (f_1 \circ f_2 \circ ... \circ f_n)(Entity_{given}, Equipment_{try}) \\
  where, \langle f_1, f_2, ..., f_n \rangle \in EquipmentConstraints_{seq}
\end{split}
\end{equation}

\subsection{Examples}

We list a few examples, in this subsection, so that we animate the
specifications above. Let's say we have a fighter. We can create an entity as
shown in \ref{eq:createent}. Translating this to source code at this point would
be trivial, since each part so far has been well defined.

\begin{equation}
\begin{split}
  \label{eq:createent}
  fighter = Entity('Arnold', Fighter, \langle str, int, dex \rangle \\
    \langle sword, shield, boots \rangle, \langle f_{strbuff}, f_{intdebuff}
  \rangle, \langle f_{eqonlyfighter} \rangle)
\end{split}
\end{equation}

\subsection{Battle System}

In this section we describe how the Entity structure (section
\ref{sec:entities}), may be used in order to form a battle system.
