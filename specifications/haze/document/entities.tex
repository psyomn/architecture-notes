\section{Entities}
\label{sec:entities}

We will now discuss about a possible organization for the data that an entity
mahy hold and manipulate. Essentially an entity should be composed of `hard
facts', and `modular facts'. Hard facts of an entity would be things such as the
entitie's \textit{strength}, \textit{magic}, and other similar information. Such
statistical information should be represented as real numbers: $\mathbb{R}$.
When it is time for implementation, you might wish to prefer \texttt{float}
types instead of \texttt{double}, as the extended precision of the latter is not
required. Each of these attributes, come bundled with a label, which will be
used to tell the player what kind of attribute the player is observing. We can
therefore express this as seen in the algebraic specifications
\ref{eq:attribute}, and \ref{eq:entity}.

\begin{equation}
\begin{split}
  \label{eq:attribute}
  Label\colon String \\
  Attribute\colon Label \times \mathbb{R}
\end{split}
\end{equation}

Now let us describe the module-plugin organization. Since we want to exhibit
different behaviors in the stats of an Entity, we want the Entity to maintain
a list of these behaviours. Notice that these behaviors take in a type
\texttt{Entity}, and map it to an \texttt{Entity} as well. So these behaviors
should essentially alter the state of the \texttt{Entity}, when fired up. So for
instance, you could create different functions for a \textit{Fighter}, and
define the attributes affected on a level up. This does not exclude the
possibility of adding more functions, and composing a function out of different
behaviours. For example, if the player is cursed, there could be a function that
gives a penalty per level up. We show such an organization in
Specifications~\ref{eq:funcmodules}, and later how it is bound to the entity in
\ref{eq:entity}.

\begin{equation}
\begin{split}
  \label{eq:funcmodules}
  f \colon Entity \to Entity \\
  MODULES_{seq}\colon \langle f_1, f_2, ..., f_n \rangle
\end{split}
\end{equation}

Another essential aspect to RPG's is the equipment system, which allows the
player to tweak the performance of the \texttt{Entity}. This is simply a type,
which holds a label, and any number of \texttt{Attribute}s. This way we can
specify modifiers for things like strength, intelligence, and more, as well as
add other things like elemental damage. We can see this in specification
\ref{eq:equipment}.

\begin{equation}
\begin{split}
  \label{eq:equipment}
  Equipment\colon Label \times \langle Attribute \rangle
\end{split}
\end{equation}

\begin{equation}
\begin{split}
  \label{eq:entity}
  EntityClass = \{ Fighter, Mage, Thief, Assassin, Archer, ... \} \\
  Entity\colon Label \times
    EntityClass \times
    \langle Attribute \rangle \times
    \langle Equipment \rangle \times MODULE_{seq}
\end{split}
\end{equation}

