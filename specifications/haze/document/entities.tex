\section{Entities}
\label{sec:entities}

We will now discuss about a possible organization for the data that an entity
mahy hold and manipulate. Essentially an entity should be composed of `hard
facts', and `modular facts'. Hard facts of an entity would be things such as the
entitie's \textit{strength}, \textit{magic}, and other similar information. Such
statistical information should be represented as real numbers: $\mathbb{R}$.
When it is time for implementation, you might wish to prefer \texttt{float}
types instead of \texttt{double}, as the extended precision of the latter is not
required. Each of these attributes, come bundled with a label, which will be
used to tell the player what kind of attribute the player is observing. We can
therefore express this tuple as seen in specifications \ref{eq:attribute}, and
\ref{eq:entity}.

\begin{equation}
\begin{split}
  f \colon Entity \to Entity \\
  MODULES_{seq}\colon \langle f_1, f_2, ..., f_n \rangle
\end{split}
\end{equation}

\begin{equation}
\begin{split}
  \label{eq:attribute}
  Label\colon String \\
  Attribute\colon Label \times \mathbb{R}
\end{split}
\end{equation}

\begin{equation}
  \label{eq:entity}
  Entity\colon Attribute \times MODULE_{seq}
\end{equation}

\begin{description}
  \itemit {potato} potato
\end{description}
