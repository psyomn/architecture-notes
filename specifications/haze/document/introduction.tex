\section{Introduction}

When I was young, I was very into video-games. At some point at the age of 12,
and 13 I was considering a career in the gaming industry, and had already
started using some very simple tools in order to make games. There were three
I had the pleasure fiddling around with. One was called Game Maker, the other 
RPG Toolkit, and the last RPG Maker. All were extremely nice tools to use and 
provided the user with satisfaction to see their game running. But none of them
were free. 

After a dozen of years, I have been toying around with my own applications to 
try and write a simple game. A recurring pattern that I observed was that many
times data would be hard coded on my end, in said games. This was not good, and
I was already thinking about alternative ways in decoupling the data from the
system. I later on found out that my thoughts were nothing new and that data 
driven systems / game engines have been around for a while. Still, I found this
to be a great exercise for the common software developer, to see how much a 
system might be decoupled, and how much modularity could be pushed. 

This finally brought me on thinking of a game maker that will be open sourced,
requirements, architecture and constraints that would have to be considered. In
the time of writing, I won't have too much time to complete this, as I already
have a lot of things on my hands (job, school, overtimes, and other, etc), but
I'd like to keep my notes here so that in the future I could come back to them
and continue from there if I'm given the chance - or if someone else can
continue this work, that would be great too. Onwards to Software Engineering.

Ultimately it would be nice to see an open sourced game maker, bundled with the
engine, and ready to ship retro games made by anyone, anywhere.

\subsection{Motivation (Userbase)} 

A primary motivation for making this engine and game editor on the user side,
would be that many have the desire to work on a videogame but do not have the
resources. Another fact that would make this software package desirable is that
if there are requests for feature implementation in the engine, then since the 
software is open source, they would be easilly implemented. 

\subsection{Motivation (Developer)}

A lot of technical implications exist in game development, and in our case we
wish to provide the tool on many different platforms. These aspects are all
something that can contribute to the technical growth of any contributing
developer.

