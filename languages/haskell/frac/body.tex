\section{Introduction}

This is my coursework for COMP6411, on the Haskell programming language. Inside
this part we discus the different powers of the programming language, the
flexibilities, and other shortcommings.  

The application in question is a fractal generator. In order to provide a
solution to this problem, we need to decide on two things:

\begin{enumerate}
\item Decide on what fractal algorithm we will choose
\item Decide on how we will display the outcome
\end{enumerate}

This clearly separates this case study into two parts. One part is to define
a library that we can call in order to generate images. The other part is to
implement the actual fractal algorithm, and auxiliary functions.

\section{Language Exploration}

In this section we show aspects of the programming language. We show the typing
system and signatures, functions, higher order functions, and recursive
functions. We also show how some basic \textit{Haskell} applications or modules
are written and used.

\subsection{Types}
We discus some basic \textit{Haskell} types in this section. The operations
listed were what were deemed useful for the case study. There exists many
other possible operations that a user of the language can use, and it is left
for the user for exploration.

\paragraph{Lists} are used by ensaring elements between square brackets. For
example, [1, 10, 2, 4] is a valid list. Lists are required to contain elements
of the same type however. For example [1, 2.0, "cat"] would raise compilation
issues. We are provided with the \textit{head} and \textit{tail} operations in 
order to manipulate lists. We are also given some very useful operations called
\textit{take} and \textit{drop} which returns the first \textit{n} specified 
elements, or returns the tail after \textit{n} elements respectively. Mapping
functionality exists, so applying lambdas on lists is also possible. Another
useful operator is the \textit{!!} infix operator, which returns us an element
at a given index. Some sample output is provided in Listing \ref{lst:listout}.
There exists a \textit{cons} operator, and in \textit{Haskell} it is the colon.

\haskellcode{lst:listout}{Simple List Operations}{src/misc/output.txt}

Another interesting feature in the lists of \textit{Haskell} is that we are
able to define ranges, and evaluate them lazily. We use two periods between two
numbers in order to achieve this. The following are valid ranges: [1..100],
[0.5..0.9]. If we omitt the rightmost number, then we tell \textit{Haskell}
that it is a list that goes on to infinity. This is legal due to the lazy
nature.

Another very useful aspect of the language is that we can use \textit{list 
comprehensions}. List comprehensions and lazy evaluation add to the power of 
the language dramatically.

\paragraph{Strings} can be specified by using the quotation marks. For example
``My kitten is cute" is a valid string. Strings however are lists of characters.
They have been aliased as the type \textit{String}. So in fact, we can say $
String = [Char]$. It follows that any operation that we can perform on a list
as previously stated, is possible to be used in Strings as well. For example, 
the head of the string ``hello" would return the character `h', and the tail
would return ``ello''. In this respect, we can also map functions onto strings.
Listing \ref{lst:strings-are-char-lsts}.

\haskellcode{lst:strings-are-char-lsts}%
            {Strings are Character Lists (Don't believe what they told you)}%
            {src/misc/strings-lst-chars.txt}

\paragraph{Others} We also have other primitives such as Booleans, 
Integers, Doubles, and tuples in our disposal. Tuples are defined by using 
parentheses, and separating its types with commas: (String, Int, Double) is a 
valid tuple. 

\subsection{Recursion}

Recursion is possible in \textit{Haskell}. Similarly to \textit{Prolog} we are
given to separate the base cases from the body of the function. The recursive
cases remain in the body of said function. Therefore, the base cases are
evaluated in the order they are appear in. For example consider the code in
Listing \ref{lst:order-of-things}. 

\haskellcode{lst:order-of-things}%
            {Cases using argument values}{src/misc/mytest.hs}

If we executed \textit{mycase 1} we would return the value `10'. If we executed
\textit{mycase 2} we would return the value `20'. If we executed this function
with the argument value `4', then we would get an error.

Now provided the mechanism on how to define base cases, let us take a look how
a factorial implementation would look like in \textit{Haskell}. Listing
\ref{lst:factorial} shows us this.

\haskellcode{lst:factorial}{Factorial in Haskell}{src/misc/Factorial.hs}

Listing \ref{lst:factorial-out} shows us some example usage.

\poutput{lst:factorial-out}{Factorial Output}{src/misc/factorial-out.txt}

One might notice that there is no safeguard for giving negative integers. This
is prevented actually on compile time.

\subsection{Visibility \& Signatures} 

In comparison to other languages such as \textit{C, C++, OCaml,} Haskell does
not define its interfaces to its modules in a separate file - there is no
separation between specification and implementation. In order to achieve
visibility, we need to specify the function names which we want to expose
between brackets after the module name. We can see such an example in Listing
\ref{lst:mymodulesimple}.

%% TODO check positioning
Haskell is a strong typed language. This adds to the compiler's task at seeing
if the user of the language is trying to do something erroneous. As such, 
\textit{Haskell} does not support casting, or other features that might cause
problems during runtime. The idea is to catch as many bugs as possible during
compile time, using the type system.

\haskellcode{lst:mymodulesimple}%
            {Exposing Functions in a Module}%
            {src/misc/MyModuleExposeName.hs}

Signatures exist in Haskell. They can be thought of as prototype functions in
\textit{C/C++}. A Haskell signature is similar to those of \textit{OCaml}. We
need to specify a name, then add a double colon, and separate it with its
type signature which contains the return type as the last type. Here is an 
example:

$$ functionName :: \langle signature \rangle $$

We can see a more implemented example in Listing \ref{lst:mymodule}. We can
observe the function name \textit{toStrList}, on its right the separator `::',
and then the type signature \textit{[Int] -$\rangle$ [String]}. Recall that we
said that the last type is the return type. Therefore this makes us understand
that all the types right before the rightmost are the inputed parameters. So
in the case of \textit{toStrList}, we understand that this function takes in
a List of integers, and returns a list of strings.

\haskellcode{lst:mymodule}%
            {Example Module with Function}%
            {src/misc/MyModule.hs}

\subsubsection{Generics}
Generics can be thought of like templates in \textit{C/C++ or Java}. This means
that the possibility of reusing a function with many types is possible. Let us
revisit the example where we were converting integers into lists (Listing
\ref{lst:mymodule}). 
This is a nice function, but it only works for \textit{Integer} input. What if
we wanted to use it for various data types? This is where the use of generics
are deemed useful. A generic is defined by a lowercase character. So if we want
a list of `things', whatever they may be, we denote that this way: $[a]$.
Extending on the previous example, we first need to change the type signature.
Since we want any type and not only Integers, we need to replace the parameter.
So the signature now becomes: $[a] \rightarrow [String]$.

However there is one last thing that we need to take care about our generic
type `a'. If we wish for the possibility of converting each element `a' into a
string, we need to impose the type constraint \textit{Show}. This requires that
any type we pass to this new function should have behavior defined for `show'
(which is similar to toString functions in other programming languages). The
signature now becomes: $(Show\ a) =\rangle [a] \rightarrow [String] $. 

As you might have inferred, constraints are imposed by writing them at the
left-most of the type signature. If more constraints were to be added, you
would separate them by commas. So for example if we also wanted the generic
type `a' to be constrained by `Ord' (if something is order-able), we would
change the constraint to: $(Show\ a,\ Ord\ a)$. Listing \ref{lst:generic} shows
the generic implementation of the list to string function, and
\ref{lst:generic-out} shows the output when providing different lists of
different element types.

\haskellcode{lst:generic}{Generic toStr}{src/misc/MyModuleGeneric.hs}

\poutput{lst:generic-out}{Generic Output}{src/misc/gen-output.txt}

\subsubsection{Higher Order Functions}

Higher order functions are supported in \textit{Haskell}. One thing to note is
how to express them inside the function you are defining, that will use that
function. The rest is not too hard to get working. For the sake of this example
we will be implementing our own filter. (Recall that a filter tests the list's
elements against a predicate, and are added to the new list if they satisfy
said predicate).

First we need to define the signature. We know there are two inputs: the
predicate, and the list to test it against. We expect a list, empty or
non-empty after the operation has completed. We can first piece together the
following: $ function \rightarrow [a] \rightarrow [a] $. That is we expect
after passing the function and a list of elements type `a', to return a list of
type element `a'. 

Next we need to define the function in more detail. To specify a higher order
function we need to encase the signature in parentheses. Our predicate gets an
element `a', and returns a `Boolean'. Therefore the predicate inside the function
we are trying to define looks like $(a \rightarrow Bool)$.

Putting the signature together we have the following: $myfilter :: (a
\rightarrow Bool) \rightarrow [a] \rightarrow [a]$. The implementation uses
recursion in order to build the list, depending on whether the predicate is
satisfied or not. Listing \ref{lst:myfilter} shows the final implementation of
the function.

\haskellcode{lst:myfilter}{Our Own Filter}{src/misc/MyFilter.hs}

You may notice the underscore in the base case of this function. That is an
anonymous variable. Since we do not care about the predicate in the base case
we are free to omitt it.

\subsection{Defining your own Infix Operators}

You can also define your own infix operators. This is a useful feature in the
programming language. Here is an example where we used this in order to
implement the `99 bottles of beer' problem. Notice the backtics on `bottles' in
Listing \ref{lst:beer}.

\haskellcode{lst:beer}{99 Bottles of Beer in Haskell}{src/misc/99bottles.hs}

\section{Problem Dissection}

For this case study, I decided to write up a fractal generator. This required
two steps. Writing a small library for outputing images, and writing the 
fractal mathematical formulas. 

\subsection{Writing the Image Library}

I was looking for a specification of an image format that would not be too hard
to implement due to the time constraint I had for this project. Therefore the
ideal format for my application would be something that could be represented as
text. Apart from being really easy to implement, it would be easy to debug as
well, since the output would be text files. 

Searching on the Internet I finally stumbled upon the \textit{PPM} format for 
images. There are two versions of the format one may implement. This is read
by the image viewer on the file headers. If you specify P6, then it expects to
read the image data as raw bits. However if you specify P3, you can enter the
image data by 3-tuples of pixels (where the tuple contains the luminoscity of
the Red, Green, and Blue colors). More information on the file format can be
read about here: \url{http://netpbm.sourceforge.net/doc/ppm.html}.

The best approach for this was to define a data type \textit{Pixel} which 
contained these three colors. Then we could define auxiliary functions to 
modify the information, or read it. Such behavior is found in Listing
\ref{lst:all:ppmpackage}. Notice first the two types: Pixel, and PPMImage. The
pixel defines another 3 elements which will be placeholders for the RGB colors,
and PPMImage, is the whole image - the header information, dimensions and image
data included. Using these two types, we build the foundations for completing
the rest of the application.

We can notice some of the auxiliary functions \textit{redOf, blueOf, greenOf}. 
Those are used in order to extract the luminoscity of the pixels. Notice how
the anonymous variables are used in order to only get the data that we want for
each. 

The next important functions are \textit{makeRow, makeImage, and
makeImageData}. These functions help us create the image in question. The
\textit{makeRow} function creates a row of pixels. This function is used `n'
times in the back-end of the \textit{makeImage} function in order to give
height to the image. We have yet more auxiliary functions such as
\textit{makeWhiteImage, makeGreenImage, makeBlackImage, etc} which provide us
shorthands to create images of x, y dimensions. The \textit{makeBlackImage} 
function is used later in order to provide a background to the fractal. 

Last but not least are the string conversion functions that return the image
as a string. This is important because as previously stated, we wish to export
the image into a text format that is viewable as a picture by an image viewer.

First we need to take care of the finely grained issue - converting a single 
pixel into a tuple of three numbers as a string. This is observed in the
\textit{pixelString} function in Listing \ref{lst:all:ppmpackage}. Then we 
define a recursive function that goes through a whole row in an image, and 
converts it into a string - namely \textit{outputRow}. Finally we define one
last function to take care of output for a whole image, \textit{imageString, 
imageStringBackend}. This concludes the requirements of having a library that
is able to generate our images.

\subsection{Julia Sets and Fractal Implementation}

The next important listing to look at is \ref{lst:all:simple}. We wish to
investigate Julia sets using the quadratic polynomial below.  The quadratic
polynomial was obtained from here:
\url{http://en.wikipedia.org/wiki/Mandelbrot\_set}. The `z' values are always
replaced by the previous value that was obtained calculating the function with
constant `c'. Constant `c' is a arbitrary imaginary number that is supplied.

$$ z_{n+1} = z_{n}^{2} + c $$ 

Now there are two things that we need to consider: the ranges that the set is
defined inside, and the range of the image dimensions. That is, we need to
super impose the dimensions of the cartesian plane where the fractal is
defined, on an array representing an image. Therefore, for each array index
that we have in our image, we stretch the range of the cartesian plane over it.
This means that each index adds a small fraction to the dimension. If for
example we had an image of size 2000 pixels wide, and our cartesian plane was
from the range -2 to 2, then for each pixel, we need to add a small amount that
is calculated in the `grain' function in Listing \ref{lst:all:simple}. This 
function translates to the following: 

$$ grain = \frac{abs(minv) + abs(maxv)}{distance_{minmax}}  $$

A list comprehension is used in order to poppulate the array with these values.
The list comprehension can be spotted inside the function \textit{makePlane}.

Another aspect to take into consideration is that we need to calculate values
using imaginary numbers. To do this, we import the complex numbers in Haskell.
We can define imaginary numbers like this: \textit{1.0 :+ 2.0}, where the left
part is the real part, and the right part is the imaginary part. Operators are
overloaded in order to perform calculations if needed. 

The plane is poppulated with these imaginary values. The real parts are
affected from left to right, by the grains as stated (so for each succeeding
pixel from left to right, we add `n' times this grain). The imaginary parts are
affected form top to bottom (using the +y to -y values for the grain formula).

Finally we need to apply the polynomial function \textit{z} recursively on a
pixel's imaginary values. The formula is repeatedly called, using the previous
output as its input. Each time, we test the Euclidean distance from the center
of the plane, to the new coordinate. The Euclidean distance uses the raw
magnitudes of the complex numbers for the `x' and `y' inputs. 

$$ euclidean = \sqrt{abs(magnitude_{real})^{2} + abs(magnitude_{imaginary})^{2}} $$

We keep track at how many times we applied the \textit{z} function until we got
a value more than 2, when calculating its Euclidean distance. If it's greater
than 2, then it means that it has escaped the plane of interest and that that
point does not belong to the Julia set. However, if it takes a considerable
amount of iterations in order to escape this boundary, we note this point as
belonging to the Julia set. The points belonging to the Julia set are what make
up the visible part of the fractal on pictures. Also, the intensity of that 
particular point's color is determined by the amount of iterations required in
order for it to escape (or remain) in the plane. 

\subsection{Image Output}

Here are some fractals, generated by the program, providing different constants
`c'. Note that the colors were edited after the output, for facilitating
printing.

\psyfig{fig:frac1}{Fractal 1}{fig/frac1.png}{6in}
\psyfig{fig:frac2}{Fractal 2}{fig/frac2.png}{6in}
\psyfig{fig:frac3}{Fractal 3}{fig/frac3.png}{6in}
\psyfig{fig:frac4}{Fractal 4}{fig/frac4.png}{6in}
\psyfig{fig:frac5}{Fractal 5}{fig/frac5.png}{6in}
\psyfig{fig:frac6}{Fractal 6}{fig/frac6.png}{6in}

