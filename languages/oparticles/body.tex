\section{Summary}
This is my coursework for SOEN6411, on the OCaml programming language. Inside
this part we discus the different powers of the programming language, the
flexibilities, and other shortcommings.  The application in question is a
particle simulator. The particle simulator defines two aspects: the bodies that
exist in the simulation, and possible obstructions. The obstructions are either
the bodies themselves, or constraints in the form of walls. The bodies are
programmed to have different velocities, acceleration, and gravitational
attraction. Calculations are made to find the 3d angle to which they travel
when they bounce off such surfaces.

\section{Language Exploration}
In this section we show some of the language features providede by OCaml.

\subsection{Interface and Implementation}

In OCaml, we are able to separate implementation from interfaces. For example,
when implement- ing the Particle class, we need to define the module’s
signatures in one file with *.mli extension, and the specify the implementation
in the respective *.ml file. This is very similar to how you would define
header files in C/C++ programs. You can refer to Listings
\ref{lst:all:particle}, and \ref{lst:all:particlei} to see the implementations
and specifications respectively, of the particle class.

\subsection{Modules}

Modules act as namespaces. Functions that are defined inside a module
definition can be accessed from other parts of the OCaml application. If they
are omitted from the module interface, then the methods are not visible and
available for local use of the module. The module must have the same name as
the file. To access another module, one needs to do this by adding \textit{open
ModuleName} at the file of interest. Listing \ref{lst:all:coordinatehelperi}
shows a module signature (interface), and \ref{lst:all:coordinatehelper} shows
the implementation
of the specification.

\subsection{Types}
There exists the basic types in OCaml such as integers, floats, booleans, strings, and lists. Here is
a small list of possible operations on each type listed above:

\begin{itemize}

\item{Integers}
\begin{itemize}
\item 1 + 1 returns 2
\item 1 - 1 returns 0
\item 5 * 2 returns 10
\item 10 / 3 returns 3 (of type integer)\end{itemize}

\item{Floats} \textit{(notice that each operation requires a period after each symbol)}.
\begin{itemize}
\item 1.0 +. 2.0 returns 3.0
\item 1.0 /. 3.0 returns 0.333333333333333315
\item 1.0 *. 4.0 returns 4.0
\item 1.0 -. 3.0 returns -2.0 \end{itemize}

\item{Booleans}\begin{itemize}
\item true \&\& true returns true
\item true $\vert\vert$ false returns true \end{itemize}

\item{Lists} \begin{itemize}
\item 1 :: [2; 3; 4];; returns [1; 2; 3; 4]
\item List.hd [1; 2; 3; 4];; returns 1
\item List.tl [1; 2; 3; 4];; returns [2; 3; 4]
\end{itemize}
\end{itemize}

\subsection{Tuples}

Tuples are supported in Ocaml. In order to define a signature for a tuple, we
chain the types we want the tuple to contain with the ‘*’ symbol. For example
if we wanted a tuple with the first two types of integer and the last one of
type float we would define it as such: \textit{int * int * float}. A practictal
example where this has been used in specification is in Listing
\ref{lst:all:coordinatei}, and its implementation in
\ref{lst:all:coordinate}.

\subsection{Object Orientation}

We are able to define classes in OCaml. To do this, we need to define the
interface (mli) file, and then implement it in the implementation file (ml).
Let us take the class Particle for this example. We need it to have an
interface that requires the implementation of features: mass, velocity,
acceleration, group, label, angle-xy, angle-xz, and its coordinates. We also
need mutators for these attributes. Listing \ref{lst:all:particlei} demonstrates
the interface, and Listing \ref{lst:all:particle} demonstrates the
implementation of the specification.

It should be noted that if the class needs to call its own methods, then in the
implementation file, we need to specify object(self) as opposed to simply
object. By doing this, we publish a class pointer within the instance of the
class, and can call methods through it (exactly like this). A quick example of
how this is used can be viewed in Listing \ref{lst:all:particle}, in the
\textit{method\_start()} where the instance calls its own method via the
expression: \textit{self \# start\_backend()}.

Something to observe is that when defining classes, we need to use the keyword
\textit{method } in order to define class methods, as opposed to using \textit{let}.

Another minor note is that method visibility is possible by the reserved word
\textit{private}. The order of the keyword `private' is important, else we get a
compilation error. An example of this can be see inside the room class, in 
listing \ref{lst:all:room}, inside the \textit{start\_backend()} method.

\subsection{Recursion}
OCaml supports recursion. However we need to explicitly specify when this is
the case, by adding a rec token after our let token. There’s two techniques to
invoke recursion. In a module, recursion is achieved by simply writing the
function in the form of: \textit{let rec funcname args = conditional $\langle
true \rangle$ {base-case}$^{+}$$\langle$false$\rangle$
\{recursive-case\}$^{+}$}. Listing \ref{lst:all:particlebuilder} shows us this,
notably in function create list.  

To define such behavior in a class instance, we need to define a function
inside a function. The inner function is the recursive function - the outer one
is simply the interfacing one. Class room (Listing \ref{lst:all:room}) shows
such implementation details in the function \textit{start\_backend()}.

