\section{Introduction}

My experience has made me see that managing a database on a running
environment, as well as in development is tedious and error prone.  Ultimately
it would be great to have a centralized, and specialized module to take care of
these aspects, without external intervention where this could apply. That is,
the concern of having a schema to date for the application in question, may be
self contained by that very application.

When decoupling this concern from applications that may benefit from such a
design and implementation we must note two important points before approaching
the problem in this manner:

\begin{itemize}
\item
If the application \textit{must} load the whole database configuration in one
shot (quicker)
\item
If the application \textit{must} build the database configuration sequentially
(slower, but provides application version management)
\end{itemize}

Both points are backed up by two valid use cases: we are concerned about
applications that might need to create the schema once, and applications that
might require a sequential upgrade. The former could be triggered when an
application runs for the first time. The latter would prove useful when the
application is switching from one version to another (note that due to the
versioning system, we are able to roll-back changes if required)

This document wishes to demonstrate the possibility of such an architectural
design. In this document, analysis and architectural diagrams will be presented
to demonstrate the rationale of the solution to the problem statement. You can
consult the table of contents, list of figures, and list of listings for any
specific, required information. 

At the last section of this case study, we will iterate once again over the
benefits of using such an organization of resources, as well as reasoning why
such an approach might not be optimal.

\subsection{Where did the idea come from?}

I was browsing some code with a friend, and we were making snide remarks on
some of the ways things were done. There was something that caught my eye: the
way they actually handled database versioning was not bad. It was not perfect
in my opinion however. And since then I have been thinking of different
approaches that could handle database versioning (and now keeping my notes in
Tex).

\subsection{Where is the idea going to?}

Ultimately I wish this to be a little more sophisticated in the future if I
have time to think more about the way I'm doing things as describe in this 
document. It would be a very nice thing if it is possible to provide a library,
or architecture other developers can use in order to supplement their needs. It
would be even better if they could in turn chip in, or give a shout about what
aspect they like or dislike about this approach in order to provide the best
for the innocent bystanders.

